%%%%%%%%%%%%%%%%%%%%%%%%%%%%%%%%%%%%%%%%%%%%%%%%%%%%%%%%%%%%%%%%%%%%%%%%%%%%%%%%
%2345678901234567890123456789012345678901234567890123456789012345678901234567890
%        1         2         3         4         5         6         7         8

\documentclass[letterpaper, 10 pt, conference]{ieeeconf}  % Comment this line out if you need a4paper

%\documentclass[a4paper, 10pt, conference]{ieeeconf}      % Use this line for a4 paper

\IEEEoverridecommandlockouts                              % This command is only needed if 
                                                          % you want to use the \thanks command

\overrideIEEEmargins                                      % Needed to meet printer requirements.

%In case you encounter the following error:
%Error 1010 The PDF file may be corrupt (unable to open PDF file) OR
%Error 1000 An error occurred while parsing a contents stream. Unable to analyze the PDF file.
%This is a known problem with pdfLaTeX conversion filter. The file cannot be opened with acrobat reader
%Please use one of the alternatives below to circumvent this error by uncommenting one or the other
%\pdfobjcompresslevel=0
%\pdfminorversion=4

% See the \addtolength command later in the file to balance the column lengths
% on the last page of the document

% The following packages can be found on http:\\www.ctan.org
\usepackage{graphics} % for pdf, bitmapped graphics files
\usepackage{epsfig} % for postscript graphics files
\usepackage{mathptmx} % assumes new font selection scheme installed
\usepackage{times} % assumes new font selection scheme installed
\usepackage{amsmath} % assumes amsmath package installed
\usepackage{amssymb}  % assumes amsmath package installed
\usepackage{cite}
\usepackage{amsfonts}
\usepackage{mathtools}
\usepackage{algorithmic}
\usepackage{graphicx}
\usepackage{textcomp}
\usepackage{xcolor}
\usepackage{tikz}
\usepackage{caption}
\usepackage{subfig}
% \usepackage{subcaption} % do not work with bibtex ?!
\usepackage{multicol}
\usepackage{pdfpages}


\title{\LARGE \bf
Title*
}

\author{P. Nom$^{1}$, P. Nom$^{1}$, P. Nom$^{2}$ and P. Nom$^{1}$ % <-this % stops a space
\thanks{*This work was supported by XXX project under grant XXX.}% <-this % stops a space
%
\thanks{$^{1}$P. Nom, P. Nom and P. Nom are with CNRS, LAAS,
        7 avenue du colonel Roche, F-31400 Toulouse, France and
        Univ. de Toulouse, UPS, LAAS, F-31400, Toulouse, France
        {\tt\small \{prenom.nom,prenom.nom,prenom.nom\}@laas.fr}}%
\thanks{$^{2}$P. Nom is with INRAE, TSCF,  9 avenue Blaise Pascal, 
        CS 20085, 63178 Aubière, France 
        {\tt\small prenom.nom@inrae.fr}}
}


\begin{document}

\maketitle
\thispagestyle{empty}
\pagestyle{empty}


%%%%%%%%%%%%%%%%%%%%%%%%%%%%%%%%%%%%%%%%%%%%%%%%%%%%%%


\begin{abstract}

\end{abstract}

% \begin{IEEEkeywords}
% \end{IEEEkeywords}

% steps for kickass intro:
% 1. describe situation in a way everyone can relate
% 2. based on situation, introduce problem
% hidden step 3: ask the question "how to solve that problem"?
% step four: give solution (or entry to solution)

\section{Introduction}

\subsection{Related works}

\subsection{Methodology}


%%%%%%%%%%%%%%%%%%%%%%%%%%%%%%%%%%%%%%%%%%%%%%%%%%%%%% System definition
\section{System definition} \label{System definition}

%%%%%%%%%%%%%%%%%%%%%%%%%%%%%%%%%%%%%%%%%%%%%%%%%%%%%% Main
\section{Main} \label{Main}

%%%%%%%%%%%%%%%%%%%%%%%%%%%%%%%%%%%%%%%%%%%%%%%%%%%%%% Experimentation
\section{Experimentation} \label{Experimentation}

%%%%%%%%%%%%%%%%%%%%%%%%%%%%%%%%%%%%%%%%%%%%%%%%%%%%%% Conclusion
\section{Conclusion} \label{Conclusion}


% Bilan : Rappeler l’objectif puis le travail effectué en mettant en lumière les contributions.
% Perspectives : propositions d’amélioration des résultats obtenus et éventuellement nouvelles pistes de recherche ouvertes.
% Attention le tout doit « glisser » pour avoir une conclusion dynamique et « punchy » car c’est la dernière impression du reviewer

%%%%%%%%%%%%%%%%%%%%%%%%%%%%%%%%%%%%%%%%%%%%%%%%%%%%%% Bibliography

% References are important to the reader; therefore, each citation must be complete and correct. If at all possible, references should be commonly available publications.

\nocite{*}
\bibliographystyle{IEEEtran} % We choose the "plain" reference style
\bibliography{biblio}

% \begin{thebibliography}{99}


% \end{thebibliography}


\end{document}
